\documentclass{article}

\usepackage{amsmath, microtype, tikz, physics, gensymb, enumitem, etoolbox}
\usepackage[letterpaper, bottom=1in, top=1in, left=1in, right=1in]{geometry}

\patchcmd\subequations
 {\theparentequation\alph{equation}}
 {\subequationsformat}
 {}{}

\newcommand{\subequationsformat}{\theparentequation.\arabic{equation}}

% load after changing subequation formatting
\usepackage{hyperref}

\title{Physics Notes 7}
\date{2/15/2023}
\author{Laith}

\begin{document}
\maketitle

\section{Introduction}

$\va{T}$ is the force vector of the rope tugging on the box $M$. 
\begin{figure}[h]
    \centering
    \begin{tikzpicture}
        \draw (0, 0) -- (0, 2) -- (2, 2) node[midway, yshift=-30] {$M$} -- (2, 0) -- (0, 0);
        \draw[-stealth] (2, 2) -- (4, 4) node[right] {$\va{T}$};
        \draw[dashed] (1, 2) -- (5, 2);
        \draw (2.5, 2.5) arc (50:-12:0.5) node [midway, xshift=5, yshift=2] {$\theta$};
    \end{tikzpicture}
\end{figure}

\noindent With what acceleration should the box be 
moving so that it comes off the ground?

\newpage
\section{Notes}
\begin{itemize}
    \item Tension: is constant through the rope.
    \item It points to the inside of the rope.
\end{itemize}

\begin{figure}[h]
    \centering
    \begin{tikzpicture}
        \draw[very thick] (0, 0) -- (6, 0);
        \draw (3, 0) -- (3, -2);
        \draw (3, -2) arc (90:540:1);
        \draw[red] (4, -3) -- (4, -5);
        \draw[red] (2, -3) -- (2, -7);
        \draw[blue] (1.5, -7) -- (2.5, -7) -- (2.5, -9) node[midway, xshift=-15] {$M$} -- (1.5, -9) -- (1.5, -7);
        \draw[blue] (3.5, -5) -- (4.5, -5) -- (4.5, -6) node[midway, xshift=-15] {$m$} -- (3.5, -6) -- (3.5, -5);
    \end{tikzpicture} 
\end{figure}
\begin{subequations}
    \begin{align}
        &T-Mg = MA  &\text{Where $T$ is the force of tension.} \\
        &T-mg = ma \\ 
        &A = -a \label{eq:1} \\
        &T - Mg = M(-a) \Rightarrow T=Mg-Ma
    \end{align}
In regards to equation \ref{eq:1}, we can utilize the fact that 
the objects are connected, thus both will move with the
magnitude of acceleration:
\[ \abs{A} = \abs{a} \] 
but in opposite directions:
\[ \boxed{A = -a} \quad \mathrm{or} \quad \boxed{-A = a} \]
\end{subequations}

\section{Newton's Third Law}
``For every force, there is an equal and opposite reaction.''
\bigskip

\noindent If object $A$ exerts force $\va{F}$ on object $B$, then 
object $B$ also exerts force $-\va{F}$ on object $A$.

\newpage
\subsection{Hypothetical}
With what velocity will the boxes move?
\begin{figure}[h]
    \centering
    \begin{tikzpicture}
        \draw[-stealth] (-2, 0.5) -- (-0.1, 0.5) node[midway, yshift=10] {$P=1\,\mathrm{N}$};
        \draw (0, 0) -- (0, 1) -- (1, 1) node[midway, yshift=-15] {$m$} -- (1, 0) -- (0, 0);
        \draw (1, 0) -- (1, 2) -- (3, 2) node[midway, yshift=-30] {$M$} -- (3, 0) -- (0, 0);
    \end{tikzpicture}
\end{figure}
$P$ is the force vector acting on $m$.

Using Newton's Third Law, we know that the force $m$ exerts on $M$ will result in 
$M$ exerting a force on $m$, thus we have two forces:
\begin{subequations}
    \begin{align}
        &f_{m\rightarrow M} &\text{Force of $m$ on $M$.} \\
        &f_{M\rightarrow m} &\text{Force of $M$ on $m$.}
    \end{align}

    According to Newton's Thrid Law, these forces act opposite to each other
    with the same magnitude, meaning the have opposite directions but the same 
    absolute force, thus:
    \begin{align}
        &f_{M\rightarrow m} = -f_{m\rightarrow M} \label{thirdlaw} &\text{Using Newton's Third Law.}
    \end{align}

    We can set up a few basic equations using these forces. We know that the force on an object 
    is equal to its mass times its acceleration and also equal to the sum of forces acting 
    on it. 
    \bigskip
    
    \noindent Applying this to box $m$, we get:
    \begin{align*}
        &F = ma \\
        &F = P + f_{M\rightarrow m} 
    \end{align*}
    Which gives us:
    \begin{align}
        &P + f_{M\rightarrow m} = ma
    \end{align}
    Applying this to box $M$, we get:
    \begin{align*}
        &F = MA \\
        &F = f_{m\rightarrow M}
    \end{align*}
    Which gives us:
    \begin{align}
        &f_{m\rightarrow M} = MA
    \end{align}
    We now have a system of equations representing the scenario:
    \begin{align}
        &P + f_{M\rightarrow m} = ma \\
        &f_{m\rightarrow M} = MA 
    \end{align}
    Let us assume that these are very hard boxes,
    something like metal.

    Since the small box $m$ is pushing the bigger box $M$, $m$ 
    cannot move at an acceleration greater than that of $M$ or 
    else it would have to basically phase through $M$, which is 
    not physically possible. However, $M$ is not under the same 
    limitations assuming nothing is in front of it, which is 
    fairly obvious if you think about it. 

    If the small box $m$ were to hit the big box $M$ with a strong 
    enough force, $M$ would reach an acceleration higher than 
    that of $m$, but only for a time. This acceleration would then 
    return to 0. You can think of that force like a bat hitting a 
    baseball. 

    For purposes, we will assume that $m$ is just pushing $M$, which 
    means that the acceleration of both boxes are equal in both 
    magnitude and direction:
    \[ A = a \]
    Thus, we can set up the previous equations as:
    \begin{align}
        &P + f_{M\rightarrow m} = m(a) \label{eq:2} \\
        &f_{m\rightarrow M} = M(a) \label{eq:3}
    \end{align}
    What this all accomplishes is leaving us with only 1 unknown variable
    to solve for, acceleration $a$. Since we established that:
    \[ f_{M\rightarrow m} = -f_{m\rightarrow M} \quad (\ref{thirdlaw}) \]
    we can rewrite equation \ref{eq:2} as:
    \begin{align}
        &F + (-f_{m\rightarrow M}) = ma \\  
        \Rightarrow &P - f_{m\rightarrow M} = ma \\
        \Rightarrow &P - Ma = ma &\text{Using equation \ref{eq:3}}
        \Rightarrow &P = ma + Ma \Rightarrow P = a(m+M) \\ 
        \Rightarrow &\boxed{a = \frac{P}{m+M}}
    \end{align}
    Now, if given the values for $m$ and $M$ as well as $P$, we would be able to 
    calculate acceleration. Since we are looking for velocity, we can integrate 
    acceleration to get velocity, however it would be as a function of time $t$.
    As a theoretical answer, it works. Don't think too hard about it.

    \begin{align}
        v = \int a\,dt \\
        v = \int \frac{P}{m+M}\,dt 
    \end{align}
    We are already given $P$ to be 1 N, but since this is a hypothetical, we will assume 
    $m=5 \,\mathrm{kg}$ and $M = 10 \,\mathrm{kg}$:
    \begin{align}
        v &= \int \frac{1 \,\mathrm{N}}{5\,\mathrm{kg}+10\,\mathrm{kg}}\,dt \\
        v &= \int \frac{1 \,\mathrm{N}}{15\,\mathrm{kg}}\,dt \\
        v &= \int \frac{1 \,\mathrm{kg\cdot m/s^2}}{15\,\mathrm{kg}}\,dt
        \intertext{\centering Kilograms (kg) cancels out.}
        v &= \int \frac{1\,\mathrm{m/s^2}}{15}\,dt \\
        v &= \frac{1}{15}\,\mathrm{m/s^2} \cdot t
        \intertext{Reminder that this means our acceleration is $\frac{1}{15}\mathrm{m/s^2}$,
        and if we plug in for $t$, which will be in seconds, the $\mathrm{m/s^2}$ will cancel 
        into m/s, which are the units of velocity. We will leave out the units in the below 
        function since it is implied and it looks nicer.}
        \Rightarrow &\boxed{\mathrm{v}(t) = \frac{1}{15}t}
    \end{align}
\end{subequations}

\end{document}
