\documentclass[12pt]{article}

\usepackage{amsmath, microtype, tikz, physics, gensymb}
\usepackage[letterpaper, bottom=1in, top=1in, left=1in, right=1in]{geometry}

\patchcmd\subequations
 {\theparentequation\alph{equation}}
 {\subequationsformat}
 {}{}

\newcommand{\subequationsformat}{\theparentequation.\arabic{equation}}

\title{Physics Notes 7}
\date{2/15/2023}
\author{Laith}

\begin{document}
\maketitle

\section{Introduction}

$\va{T}$ is the force vector of the rope tugging on the box $M$. 
\begin{figure}[h]
    \centering
    \begin{tikzpicture}
        \draw (0, 0) -- (0, 2) -- (2, 2) node[midway, yshift=-30] {$M$} -- (2, 0) -- (0, 0);
        \draw[-stealth] (2, 2) -- (4, 4) node[right] {$\va{T}$};
        \draw[dashed] (1, 2) -- (5, 2);
        \draw (2.5, 2.5) arc (50:-12:0.5) node [midway, xshift=5, yshift=2] {$\theta$};
    \end{tikzpicture}
\end{figure}

\noindent With what acceleration should the box be 
moving so that it comes off the ground?

\newpage
\section{Notes}
\begin{itemize}
    \item Tension: is constant through the rope.
    \item It points to the inside of the rope.
\end{itemize}

\begin{figure}[h]
    \centering
    \begin{tikzpicture}
        \draw[very thick] (0, 0) -- (6, 0);
        \draw (3, 0) -- (3, -2);
        \draw (3, -2) arc (90:540:1);
        \draw[red] (4, -3) -- (4, -5);
        \draw[red] (2, -3) -- (2, -7);
        \draw[blue] (1.5, -7) -- (2.5, -7) -- (2.5, -9) node[midway, xshift=-15] {$M$} -- (1.5, -9) -- (1.5, -7);
        \draw[blue] (3.5, -5) -- (4.5, -5) -- (4.5, -6) node[midway, xshift=-15] {$m$} -- (3.5, -6) -- (3.5, -5);
    \end{tikzpicture} 
\end{figure}
\begin{subequations}
    \begin{align}
        &T-Mg = MA  &\text{Where $T$ is the force of tension.} \\
        &T-mg = ma \\ 
        &A = -a \label{eq:1} \\
        &T - Mg = M(-a) \Rightarrow T=Mg-Ma \\
    \end{align}
\end{subequations}
In regards to line \ref{eq:1}, we can utilize the fact that 
the objects are connected, thus both will move with the
magnitude of acceleration:
\[ \abs{A} = \abs{a} \] 
but in opposite directions:
\[ \boxed{A = -a} \quad \mathrm{or} \quad \boxed{-A = a} \]

\section{Newton's Third Law}
``For every force, there is an equal and opposite reaction.''
\bigskip

\noindent If object $A$ exerts force $\va{F}$ on object $B$, then 
object $B$ also exerts force $-\va{F}$ on object $A$.

\newpage
\subsection{Example Problem}
With what velocity will the boxes move?
\begin{figure}[h]
    \centering
    \begin{tikzpicture}
        \draw[-stealth] (-2, 0.5) -- (-0.1, 0.5) node[midway, yshift=10] {$F=1\,\mathrm{N}$};
        \draw (0, 0) -- (0, 1) -- (1, 1) node[midway, yshift=-15] {$m$} -- (1, 0) -- (0, 0);
        \draw (1, 0) -- (1, 2) -- (3, 2) node[midway, yshift=-30] {$M$} -- (3, 0) -- (0, 0);
    \end{tikzpicture}
\end{figure}
\begin{subequations}
    \begin{align}
        &F + f_{M\rightarrow m} = ma \\
        &f_{m\rightarrow M} = MA \\
    \end{align}
    Let us assume that these are very hard boxes,
    something like metal.

    Since the small box $m$ is pushing the bigger box $M$, $m$ 
    cannot move at an acceleration greater than that of $M$ or 
    else it would have to basically phase through $M$, which is 
    not physically possible. However, $M$ is not under the same 
    limitations assuming nothing is in front of it, which is 
    fairly obvious if you think about it. 

    If the small box $m$ were to hit the big box $M$ with a strong 
    enough force, $M$ would reach an acceleration higher than 
    that of $m$, but only for a time. This acceleration would then 
    return to 0.  

    For purposes, we will assume that
    \begin{align}
        &f_{M\rightarrow m} = -f_{m\rightarrow M} &\text{Using Newton's Third Law.} \\
        &f_{M\rightarrow m} = -f_{m\rightarrow M}
    \end{align}
\end{subequations}

\end{document}
