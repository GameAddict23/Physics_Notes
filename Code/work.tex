\documentclass{article}

\usepackage[letterpaper, top=1in, bottom=1in, left=1in, right=1in]{geometry}
\usepackage{
    physics, 
    amsmath, 
    enumitem, 
    nicefrac, 
    fancyhdr, 
    microtype, 
    pgfplots, 
    tikz, 
    float,
    hyperref
}

\pgfplotsset{compat=newest}
\pagestyle{fancy}

\title{Work and Spring Force}
\author{Laith Toom}
\date{21/2/2023}

\begin{document}
\maketitle
\tableofcontents 

\newpage 

\section{What is Work?}
Work, $W$, is the change in kinetic energy:
\[ W = \frac{1}{2}M\abs{\va{v_f}}^2 - \frac{1}{2}M\abs{\va{v_i}}^2 = \boxed{ \frac{M}{2} \left(\abs{\va{v_f}}^2 - \abs{\va{v_i}}^2 \right) } \]
Work is also equal to the dot product between a force vector and displacement vector:
\[ W = \abs{\va{F}}\cdot\abs{\va{r}}\cdot\cos(\theta) \]

In the case of complex trajectory, such as:
\begin{figure}[H]
    \centering
    \begin{tikzpicture}
        \draw[->] (0, 0) -- (0, 10);
        \draw[->] (0, 0) -- (10, 0);

        \draw[->, red] (0, 0) -- (3, 4) node[above left, midway] {$\va{r_i}$};
        \draw[->, red] (0, 0) -- (7, 5) node[above left, midway] {$\va{r_f}$};

        \draw[dotted] (3, 4) .. controls (5, 3) and (6, 8) .. (7, 5);

        \draw[->, blue] (3, 4) -- (3, 0) node[below] {$Mg$};
    \end{tikzpicture}
\end{figure}
then work can be calculated by taking the integral of the trajectory, where 
the trajectory will be divided into very small vectors $\mathrm{d}\va{r}$:
{\large
\begin{align}
    W &= \int_{\va{r_i}}^{\va{r_f}} \abs{\va{F}} \cdot \mathrm{d}\va{r} \\
      &= \int_{\va{r_i}}^{\va{r_f}} Mg\,\hat{y} \cdot \mathrm{d}\va{r} \\
      &= Mg\,\hat{y} \int_{\va{r_i}}^{\va{r_f}} \mathrm{d}\va{r} \\
      &= Mg\,\hat{y} \left(\va{r_f} - \va{r_i}\right)
\end{align}
}
\newpage
Subtracting the two vectors $\va{r_f}$ and $\va{r_i}$ will result in a new vector,
$\overrightarrow{\bf{r_f-r_i}}$:
\begin{figure}[H]
    \centering
    \begin{tikzpicture}
        \draw[->, red] (0, 0) -- (3, 4) node[above left, midway] {$\va{r_i}$};
        \draw[->, red] (0, 0) -- (7, 5) node[above left, midway] {$\va{r_f}$};

        \draw[->, very thick, black] (3, 4) -- (7, 5) node[above, yshift=5, midway] {$\overrightarrow{\bf{r_f-r_i}}$};

        \draw[->, blue] (3, 4) -- (3, 0) node[below] {$Mg$};
    \end{tikzpicture}
\end{figure}
and if we look closer, we will find a right triangle:
\begin{figure}[H]
    \centering
    \begin{tikzpicture}
        \draw[->, very thick, black] (3, 4) -- (7, 5) node[above, yshift=5, midway] {$\overrightarrow{\bf{r_f-r_i}}$};
        \draw[dashed] (3, 4) -- (7, 4);
        \draw[dashed] (7, 4) -- (7, 5);

        \draw (4.5, 4.39) arc(13:-10:1) node[right, midway] {$\alpha$};
        \draw (5, 4.5) arc(10:-91:2) node[right, midway] {$\theta$};

        \draw (3, 3.5) -- (3.5, 3.5) -- (3.5, 4);

        \draw[->, blue] (3, 4) -- (3, 0) node[below] {$Mg$};
    \end{tikzpicture}
\end{figure}
We see that:
\[ \alpha = \theta - 90 \Longleftrightarrow \theta = \alpha + 90 \]
Work is the dot product between the force vector and displacement vector. 
In our case, the force vector is $Mg$ and the displacement vector is 
$\overrightarrow{\bf{r_f-r_i}}$, and we see that $\theta$ is the angle 
between the two vectors, thus:
\begin{align}
    W_g = Mg \cdot \abs{\overrightarrow{\bf{r_f-r_i}}} \cdot \cos(\theta) 
    \intertext{Since $\theta=\alpha+90$, we can rewrite this as:}
    W_g = Mg \cdot \abs{\overrightarrow{\bf{r_f-r_i}}} \cdot \cos(\alpha+90)  
    \intertext{and since $\cos(A+90)=-\sin(A)$, we can rewrite this again as:}
    W_g = Mg \cdot \abs{\overrightarrow{\bf{r_f-r_i}}} \cdot -\sin(\alpha)  
\end{align}
Thus, we find the formula:
\[ \boxed{ W_g = -Mg\abs{\overrightarrow{\bf{r_f-r_i}}}\sin(\alpha) } \]

\end{document}
