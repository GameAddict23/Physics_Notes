\documentclass{article}

\usepackage{amsmath}
\usepackage{physics}
\usepackage{tikz, tikz-3dplot, pgfplots}
\usepackage{microtype}
\usepackage[letterpaper]{geometry}
\usepackage{float}

\pgfplotsset{compat=newest}

\newcommand{\dt}{\mathrm{d}t}
\newcommand{\x}{\mathrm{x}}
\newcommand{\y}{\mathrm{y}}
\newcommand{\z}{\mathrm{z}}
\newcommand{\vel}{\mathrm{v}}
\newcommand{\ddt}[1]{\frac{\mathrm{d}#1}{\dt}}
\newcommand{\dydt}{\frac{\mathrm{d}y}{\dt}}

\title{Newton's Law}
\date{2/6/2023}
\author{Laith}

\begin{document}
\maketitle

\section{Laws:}

\subsection*{First Law:}
    If an object is not experiencing the effect of any 
    force, then it will either remain stationary \textbf{or}
    keep moving with constant \underline{velocity}.

\subsection*{Second Law:}
    If force $\va{F}$ is acting on an object with 
    mass $m$, then the acceleration $\va{a}$ is given by:
    \[\va{F}=m\va{a}\]

\begin{figure}[H]
    \begin{tikzpicture}
        \draw (0, 0) -- (15, 0);

        \draw[thick] (1, 0) -- (1, 2);
        \draw[thick] (1.5, 0) -- (1.5, 1.5);
        \draw[thick] (4.5, 0) -- (4.5, 1.5);
        \draw[thick] (1.5, 1.5) -- (4.5, 1.5);
        \draw[thick] (1, 0) -- (1.5, 0);
        \draw[thick] (4.5, 0) -- (5, 0);
        \draw[thick] (5, 0) -- (5, 2);
        \draw[thick] (1, 2) -- (5, 2);
        \draw (2, 2) -- (2, 3);
        \draw (4, 2) -- (4, 3);
        \draw (2, 3) -- (4, 3);

        \draw[blue, stealth-stealth] (2.5, 0) node[below] {$m_g$} -- (2.5, 4) node[above] {$N$};

        \node at (3, 2.5) {$m$};
    \end{tikzpicture} 
\end{figure}

% You are standing distance $d=2\mathrm{m}$ from your neighbor wall.
% The window is at height $h=1\mathrm{m}$ from the ground. You can 
% kick the ball such that it starts moving with speed $v=1\mathrm{m/s}$.
% With what angle with horizon should you kick the ball so that
% it hits the window?

\end{document}