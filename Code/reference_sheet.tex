\documentclass{article}

\usepackage{microtype, gensymb, physics, amsmath}
\usepackage[letterpaper, top=1in, bottom=1in, left=1in, right=1in]{geometry}

\begin{document}
\section{Equations}
\subsection{Constant Acceleration Kinematic Equations}
Determine which equation to use based off which variable 
is not given:
\begin{align*}
    &\text{\underline{Constant Acceleration}} &\text{\underline{Not Given}} 
    \\
    &v = v_0 + at                      &\Delta{x}        \\
    &\Delta{x} = v_0t + \frac{1}{2}at^2 &v                \\
    &v^2 = {v_0}^2 + 2a\Delta{x}     &t                \\
    &\Delta{x} = \frac{1}{2}(v_0+v)t   &a                \\
    &\Delta{x} = vt - \frac{1}{2}at^2  &v_0
\end{align*}

\subsection{Special Case Equations}
If an object only experiences normal force and gravity on an inclined plane of angle $\theta$:
\[ a = -g\sin(\theta) \]        

\bigskip 

\noindent In a pulley system of two objects, the acceleration between the objects relate with the equation:
\[ a = -A \]

\bigskip

\noindent Newton's Second Law states:
\begin{align*}
    &\sum \va{F} = m\va{a} \\[2ex]
    &\sum F_x = m\,a_x \\[2ex]
    &\sum F_y = m\,a_y
\end{align*}

\end{document}