\documentclass{article}

\usepackage{microtype}
\usepackage{amsmath}
\usepackage{tikz}
\usepackage{enumitem}
\usepackage{etoolbox}
\usepackage[letterpaper]{geometry}

\newcommand{\diff}[1]{\frac{#1}{dt}}
\newcommand{\vel}{\mathrm{v}}
\newcommand{\x}{\mathrm{x}}
\newcommand{\inter}{\int_{t_1}^{t_2}}
\newcommand{\tvert}{\biggr\rvert_{t_1}^{t_2}}

\patchcmd\subequations
 {\theparentequation\alph{equation}}
 {\subequationsformat}
 {}{}

\newcommand{\subequationsformat}{\theparentequation.\arabic{equation}}

\author{Laith}
\title{2-5: Free-Fall Acceleration}
\date{1/29/2023}

\begin{document}

\maketitle
\section{Gravity}
Acceleration by gravity is represented by $g$. The magnitude of acceleration by
gravity is:
 \[g = 9.8\,\mathrm{m/s^2}\]
If an object is falling its Acceleration would be equal to $-g$:
    \[a = -g = -9.8\,\mathrm{m/s^2}\]
{\small\textbf{Note:} Do not substitue $-9.81\,\mathrm{m/s^2}$ for $g$.}
\paragraph*{Example}
Suppose we tossed a tomato up to the air. The velocity start off positive, but since
acceleration is negative, the magnitude of velocity (speed) will decrease until it reaches 0, at which the
tomato reached its \textbf{maximum height}. Now, the velocity of the tomato becomes 
increasingly negative as the tomato falls back to the ground.  
\begin{figure}[hh]
    \begin{center}
    \begin{tikzpicture}
        \node[circle, draw, scale=0.7] at (0.7, 5) (t) {Tomato};
        \node[left, color=blue] at (0, 2.5) (v1) {$v > 0$};
        \node[right, color=blue] at (1.4, 2.5) (v2) {$v < 0$};
        \node[above] at (0.7, 5.6) (v3) {$v = 0$};
        \node[left, color=red] at (-1.1, 2.5) (a1) {$a = -g$};
        \node[right, color=red] at (2.5, 2.5) (a2) {$a = -g$};
        
        \draw[->, color=blue] (0, 0) -- (0, 5);
        \draw[->, color=blue] (1.4, 5) -- (1.4, 0);
        \draw[<-, color=red] (-1.1, 0) -- (-1.1, 5);
        \draw[<-, color=red] (2.5, 0) -- (2.5, 5);
        \draw (-1.5, 0) -- (3, 0);
    \end{tikzpicture}
    \end{center}
\end{figure}

\subsection{Problem}
In figure 2-13, a pitcher tosses a baseball up along a y axis, with
an initial speed of 12 m/s. 
\begin{enumerate}[label=(\alph*)]
    \item How long does the ball take to reach its maximum height?
    \item What is the ball’s maximum height above its release point?
    \item How long does the ball take to reach a point 5.0 m above
its release point?
\end{enumerate}
\subsubsection{Part (a)}
To find out how long the ball takes to reach its maximum height, we need to determine
the time it takes for $v_1=12\,\mathrm{m/s}$ to reach $v_2=0\,\mathrm{m/s}$. Since we know
acceleration $a$ is equal to $-g=9.8\,\mathrm{m/s}$, we can integrate to get a function for
velocity:
\begin{subequations}    
\begin{align}
    \inter \diff{dv} &= \inter a\,dt = \inter -9.8\,dt \\
    \vel(t)\tvert &= a\cdot t\tvert = -9.8 \cdot t\tvert\\
    \vel(t)\tvert &= -9.8 \cdot t\tvert
\end{align}
We can set $t_1=0\mathrm{s}$ and $t_2=t$ where $t$ is how long the ball takes to reach its
maximum height. Since the ball was launched with an initial speed of $12\,\mathrm{m/s}$, the
time at this moment would be $0\mathrm{s}$, which is equal to $t_1$. Thus, we can say that 
$\vel(t_1)=12\mathrm{m/s}$, leaving us with the only unknown being $\vel(t_2)$, or simply $\vel(t)$.
\begin{align}
    \vel(t_2)-\vel(t_1) &= -9.8 \cdot t_2 - -9.8 \cdot t_1 \\
    \vel(t_2)-\vel(0) &= -9.8 \cdot t_2 - -9.8 \cdot 0 \\
    \vel(t_2) - 12 &= -9.8\cdot t_2 \\
    \vel(t) - 12 &= -9.8\cdot t \\
    \Rightarrow \vel(t) &= -9.8\cdot t + 12
\end{align}
Now, since we want to find time $t$ when velocity $v(t) = 0$, we can set $v(t) = 0$
and solve for $t$.
\begin{align}
    0 &= -9.8 \cdot t + 12 \\
    \Rightarrow &-12 = -9.8 \cdot t \\
    \Rightarrow &\frac{-12}{-9.8} = t \\
    t &= \frac{12}{9.8}\,\mathrm{s}=1.2\,\mathrm{s}
\end{align}
\end{subequations}
Alternatively, we could utilize the fact that we are looking at two endpoints for velocity,
thus we can use the equation for average acceleration:
\begin{subequations}
    \begin{align}
        a_{avg} &= \frac{\Delta v}{\Delta t} = \frac{v_2-v_1}{t_2-t_1} \\
        -9.8 &= \frac{0-12}{t_2-0} \\
        -9.8 &= \frac{-12}{t_2} \\
        \Rightarrow -9.8\cdot t_2 &= -12 \\
        \Rightarrow t_2 &= \frac{-12}{-9.8} \\
        t_2 &= \frac{12}{9.8} = 1.2\\
        t &= 1.2\mathrm{s}
    \end{align}
\end{subequations}
This approach is more optimal in this case, however we wouldn't be able to calculate the velocity
at any other point in time $t$.

\subsubsection{Part (b)}
Now we want to find the maximum height of the ball. Given that we have an equation for velocity, we can
integrate $v(t)$ to get displacement $x(t)$.
\begin{subequations}
    \begin{align}
        \x(t)\tvert &= \inter \vel(t) \,dt\\
        \x(t)\tvert &= \inter -9.8\cdot t + 12\,dt \\
        \x(t)\tvert &= \inter -9.8\cdot t + \inter 12\,dt \\
        \x(t)\tvert &= -9.8\cdot \frac{t^2}{2}\tvert + 12t \tvert \\
        \x(t_2)-\x(t_1) &= [-9.8\cdot \frac{(t_2)^2}{2} - -9.8\cdot \frac{(t_1)^2}{2}] + (12t_2 - 12t_1) \\
        \x(t)-\x(0) &= [-9.8\cdot \frac{t^2}{2} - -9.8\cdot \frac{(0)^2}{2}] + (12t - 12(0)) \\
        \x(t)-\x(0) &= -9.8\cdot \frac{t^2}{2} + 12t \\
    \end{align}
    Since the ball is tossed at $t_1=0$, we can say that $x(t_1)$ which equals $x(0)$ is equals $0$. We
    also know that, from \emph{part (a)}, the ball reaches its maximum height at time $t = 1.2\,\mathrm{s}$, thus all we
    need to do is substitue.
    \begin{align}
        \x(1.2) - (0) &= -9.8\cdot \frac{(1.2)^2}{2} + 12(1.2) \\
        \x(1.2) &= -9.8\cdot 0.72 + 14.4 \\
        \x(1.2) &= -9.8\cdot 0.72 + 14.4 \\
        \x(1.2) &= 7.3\,\mathrm{m}
    \end{align}
\end{subequations}

\subsubsection{Part (c)}
Now we want to find the time it takes the ball to reach $5.0\,\mathrm{m}$ above its release point.
Since we have an equation for displacement, we can simply solve for $t$ by setting $x(t)=5$.
    \begin{subequations}
    \begin{align}
        \x(t) &= -9.8\cdot \frac{t^2}{2} + 12t \\
        \x(t) &= 5 \\  
        5 &= -9.8\cdot \frac{t^2}{2} + 12t \\
        5 &= -4.9\cdot t^2 + 12t \\
        \Rightarrow 0 &= -4.9t^2 + 12t - 5
    \end{align}
    Then we use the quadratic formula:
    \[t = \frac{-b\pm \sqrt{b^2-4ac}}{2a}\] 
    \begin{align}
        &a = -4.9\\
        &b = 12\\
        &c = -5\\
        t &= \frac{-(12)\pm \sqrt{12^2-4(-4.9)(-5)}}{2(-4.9)} \\
        t &= \frac{-(12)\pm \sqrt{46}}{2(-4.9)} \\
        t &= \frac{-(12)\pm 6.78}{-9.8} \\
        t &= \{\frac{-12+6.78}{-9.8}\,, \frac{-12-6.78}{-9.8}\} \\
        t &= \{0.53\,\mathrm{s}\,, 1.9\,\mathrm{s}\}
    \end{align}
    It turns out that the ball reaches $5.0\,\mathrm{m}$ at two times, $t_1=0.53\,\mathrm{s}$
    and $t_2=1.9\,\mathrm{s}$. This makes sense since the ball will reach $5.0\,\mathrm{m}$
    when traveling upwards to its maximum height of $7.3\,\mathrm{m}$ and then once again as
    it travels downards to the ground.
    \end{subequations}

\end{document}