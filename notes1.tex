\documentclass{article}

\usepackage{microtype}
\usepackage{amsmath}

\newcommand{\diff}[1]{\frac{#1}{dt}}

\author{Laith}
\title{Introduction to Physics}
\date{1/25/2023}

\begin{document}
	\maketitle
	\section{What is Physics?}
	Physics is the study of how the objects around us change over time. 
	This includes their \textbf{position}, \textbf{displacement}, \textbf{distance}, and \textbf{speed}.
	\subsection{Position}
	The \textbf{position} of an object is its relative distance from a reference point. For example, if we had a red dot and a green dot on a line, then we could
	determine the position of the red dot to be its distance from the green dot and the position of the green dot to be its distance from the red dot.
	We can express position as a function $\mathrm{x(t)}$.
	\subsection{Displacement}
	\textbf{Displacement} is the difference between an objects new position and its previous position. If we had a red dot move from point 1 to point 2, then the
	displacement of the red dot would be the difference in position of point 2 and point 1.
	Mathematically, we can represent this as $\Delta{x}$.
	\[\Delta{x} = x_2-x_1\]
	Since position is a function, $x_2$ and $x_1$ can be written as $x(t_2)$ and $x(t_1)$, where $t_2$ is the time at which
	the object was at position 2 and $t-1$ is the time at which the object was at position 1. Thus, we can write 
	displacement as:
	\[\Delta{x} = \mathrm{x(t_2)}-\mathrm{x(t_1)}\]  
	\subsection{Distance}
	\textbf{Distance} is the length of the path traveled by an object from one point to another.
	\subsection{Speed}
	\textbf{Speed} is how fast an object moved from one point to another. As an extension, \textbf{velocity} is just speed, but in specific direction. In other words,
	speed is a \emph{scalar}, velocity is a \emph{vector}, but both represent how fast an object moves. The rate at which velocity change over time is known as \textbf{acceleration}.
	We can represent velocity as a function $\mathrm{v(t)}$. We could try to define this function as the ratio between displacement and time elapsed between two points:
	\[\overline{\mathrm{v(t)}} = \frac{\Delta{x}}{\Delta{t}}\]
	The issue with this equation is that velocity can change at any given time, thus in order to get closer to a true representation
	of velocity, we need to look at instantaneous intervals of velocity. In other words: we differentiate position.
	\[\mathrm{v(t)} = \diff{dx}\]
	By differentiating position, we will get the rate at which position changes at a specific time $t$, which is the velocity at that specific time.
	\subsection{Acceleration}
	\textbf{Acceleration} is the rate at which the velocity of an object changes, however with a few exceptions,
	we will assume acceleration to be constant for the majority of this course. Thus, we will represent
	acceleration as a variable $a$, and if acceleration is not constant, a function $\mathrm{a(t)}$.

	Since acceleration is the rate at velocity changes, we can differentiate velocity to get acceleration:
	\[\mathrm{a} = \diff{dv}\]
	This also means that if we want to get velocity, but only have acceleration, we can integrate acceleration to get
	velocity:
	\[\mathrm{v(t)\biggr\rvert_{t_1}^{t_2}} = \int_{t_2}^{t_1}a\,dt\]
	\[\mathrm{v(t)\biggr\rvert_{t_1}^{t_2}} = a\cdot t\biggr\rvert_{t_1}^{t_2}\]
	\[\mathrm{v(t)\biggr\rvert_{t_1}^{t_2}} = a\cdot t_2 - a \cdot t_1 = a\cdot (t_2-t_1)\]
	\[\mathrm{v(t_2)-v(t_1)} = a\cdot(t_2 - t_1)\]
	\section{Time}
	Objects always move with respect to time, as the definition of physics would imply. Thus, we treat time as an independant variable; while position, displacement,
	distance, speed, velocity, and acceleration are all functions of time.
	\section{Formulas}
	\[\mathrm{v(t)} = \diff{dx}\]
	\[\mathrm{a(t)} = \diff{dv}\]
\end{document}
