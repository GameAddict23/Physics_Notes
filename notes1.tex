\documentclass{article}

\usepackage{microtype}

\newcommand{\diff}[1]{\frac{#1}{dt}}

\author{Laith}
\title{Introduction to Physics}
\date{1/25/2023}

\begin{document}
	\maketitle
	\section{What is Physics?}
	Physics is the study of how the objects around us change over time. 
	This includes their \textbf{position}, \textbf{displacement}, \textbf{distance}, and \textbf{speed}.
	\subsection{Position}
	The \textbf{position} of an object is its relative distance from a reference point. For example, if we had a red dot and a green dot on a line, then we could
	determine the position of the red dot to be its distance from the green dot and the position of the green dot to be its distance from the red dot.
	We can express position as a function $\mathrm{x(t)}$.
	\subsection{Displacement}
	\textbf{Displacement} is the difference between an objects new position and its previous position. If we had a red dot move from point 1 to point 2, then the
	displacement of the red dot would be the difference in position of point 2 and point 1.
	Mathematically, we can represent this as $\Delta{x}$.
	\[\Delta{x} = x_2-x_1\]
	\subsection{Distance}
	\textbf{Distance} is the length of the path traveled by an object from one point to another.
	\subsection{Speed}
	\textbf{Speed} is how fast an object moved from one point to another. As an extension, \textbf{velocity} is just speed, but in specific direction. In other words,
	speed is a \emph{scalar}, velocity is a \emph{vector}, but both represent how fast an object moves. The rate at which velocity change over time is known as \textbf{acceleration}.
	We can represent velocity as a function $\mathrm{v(t)}$:
	\[\mathrm{v(t)} = \diff{dx}\]
	\subsection{Acceleration}
	\textbf{Acceleration} is the rate at which the velocity of an object changes, however with a few exceptions,
	we will assume acceleration to be constant for the majority of this course. Nonetheless, we can represent
	acceleration as a function $\mathrm{a(t)}$:
	\[\mathrm{a(t)} = \diff{dv}\]	
	\section{Time}
	Objects always move with respect to time, as the definition of physics would imply. Thus, we treat time as an independant variable; while position, displacement,
	distance, speed, velocity, and acceleration are all functions of time.
	\section{Formulas}
	\[\mathrm{v(t)} = \diff{dx}\]
	\[\mathrm{a(t)} = \diff{dv}\]
\end{document}
